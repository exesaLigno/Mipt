\documentclass[12pt]{article}
\usepackage{hyperref}
\usepackage[warn]{mathtext}
\usepackage[T2A]{fontenc}
\usepackage[utf8]{inputenc}
\usepackage[russian]{babel}
\usepackage{cite}
\usepackage{amsfonts}
\usepackage{lineno}
\usepackage{subfig}
\usepackage{graphicx}
\usepackage{xcolor}
\usepackage{bm}
\usepackage{graphicx}
\usepackage{amssymb}
\usepackage{hyperref}
\usepackage[left=2cm,right=2cm,top=2cm,bottom=2cm]{geometry}
\usepackage{indentfirst}
\DeclareSymbolFont{T2Aletters}{T2A}{cmr}{m}{it}

\DeclareGraphicsExtensions{.png,.jpg,.svg,.pdf}
\author{Карцев Вадим}
\date{21 марта 2022 г.}
\title{Распараллеливание алгоритма вычисления числа $e^x$}

\begin{document}

  \maketitle

  Изначально имеем формулу вичисления числа $e^x$ следующего вида:

  \begin{equation} \label{eq:default}
    e^x = 1 + \frac{x}{1!} + \frac{x^2}{2!} + \frac{x^3}{3!} + ...
  \end{equation}

  Сразу видно, что члены последовательности рекурсивно получаются друг из друга

  $$
    e^x = a_0 + a_1 + a_2 + a_3 + ...
  $$

  $$
    a_0 = 1 \hspace{1cm}
    a_{n+1} = \frac{x}{n+1} a_n
  $$

  Это является основной оптимизацией, которая позволяет решить проблему
  переполнения в рассчете факториала - будем получать следующий член ряда,
  используя предыдущий.

  Теперь разберемся с распараллеливанием алгоритма. Очевидно, обсчитывать
  факториал 4 раза будет уже неоптимально, поэтому преобразуем формулу для
  оптимизации в параллельном алгоритме. Для примера будем рассматривать два
  процесса, первый из которых будет обсчитывать сумму членов ряда с $0$ по $n$,
  а второй -- с $n+1$ по $m$, где m -- общее число членов ряда.

  \begin{equation}
    e^x = 1 + \frac{x}{1!} + \frac{x^2}{2!} + ... + \frac{x^n}{n!} +
    \left(\frac{x^{n+1}}{(n+1)!} + \frac{x^{n+2}}{(n+2)!} + ... +
    \frac{x^m}{m!}\right)
  \end{equation}

  Теперь мы можем вынести из скобки общий множитель $\frac{x^{n+1}}{(n+1)!}$.
  Этот множитель можно оптимально получать из последнего слагаемого первой
  половины ряда. Так, получим окончательно оптимальную формулу для параллельного
  рассчета числа $e^x$.

  \begin{equation} \label{eq:parallel}
    e^x = 1 + \frac{x}{1!} + \frac{x^2}{2!} + ... + \frac{x^n}{n!} +
    \frac{x^{n+1}}{(n+1)!} \left(1 + \frac{x}{n+2} + ... +
    \frac{x^{m-(n+1)}}{(n+2)(n+3)...(m-1)(m)}\right)
  \end{equation}

  Лагко заметить, что сумма до скобок и сумма в скобках вычисляется схожим
  образом, как было описано для обсчета выражения \ref{eq:default}. Именно эти
  суммы мы и заставим обсчитывать параллельные процессы. Так же легко понять,
  что сумму внутри скобки в формуле \ref{eq:parallel} можно тем же образом
  разбить на две части. Таким образом, такой способ распараллеливания можно
  распространить и на число процессов, большее двух.

  Таким образом, в каждом процессе будем считать свою часть суммы ряда, затем
  передавать в процесс с номером на единицу меньше текущего, который будет
  обсчитывать помимо своей части еще и множитель для следующей части суммы.

  Таким образом, в процессе с нулевым номером в итоге соберется конечное
  значение суммы.

\end{document}
